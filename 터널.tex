%	-------------------------------------------------------------------------------
% 
%
%
%
%
%
%
%
%
%
%	-------------------------------------------------------------------------------

	\documentclass[12pt, a4paper, oneside]{book}
%	\documentclass[12pt, a4paper, landscape, oneside]{book}

		% --------------------------------- 페이지 스타일 지정
		\usepackage{geometry}
%		\geometry{landscape=true	}
		\geometry{top 		=10em}
		\geometry{bottom		=10em}
		\geometry{left		=8em}
		\geometry{right		=8em}
		\geometry{headheight	=4em} % 머리말 설치 높이
		\geometry{headsep		=2em} % 머리말의 본문과의 띠우기 크기
		\geometry{footskip		=4em} % 꼬리말의 본문과의 띠우기 크기
% 		\geometry{showframe}
	
%		paperwidth 	= left + width + right (1)
%		paperheight 	= top + height + bottom (2)
%		width 		= textwidth (+ marginparsep + marginparwidth) (3)
%		height 		= textheight (+ headheight + headsep + footskip) (4)



		%	===================================================================
		%	package
		%	===================================================================
%			\usepackage[hangul]{kotex}				% 한글 사용
			\usepackage{kotex}					% 한글 사용
			\usepackage[unicode]{hyperref}			% 한글 하이퍼링크 사용

		% ------------------------------ 수학 수식
			\usepackage{amssymb,amsfonts,amsmath}	% 수학 수식 사용
			\usepackage{mathtools}				% amsmath 확장판

			\usepackage{scrextend}				% 
		

		% ------------------------------ LIST
			\usepackage{enumerate}			%
			\usepackage{enumitem}			%
			\usepackage{tablists}				%	수학문제의 보기 등을 표현하는데 사용
										%	tabenum


		% ------------------------------ table 
			\usepackage{longtable}			%
			\usepackage{tabularx}			%
			\usepackage{tabu}				%




		% ------------------------------ 
			\usepackage{setspace}			%
			\usepackage{booktabs}		% table
			\usepackage{color}			%
			\usepackage{multirow}			%
			\usepackage{boxedminipage}	% 미니 페이지
			\usepackage[pdftex]{graphicx}	% 그림 사용
			\usepackage[final]{pdfpages}		% pdf 사용
			\usepackage{framed}			% pdf 사용

			
			\usepackage{fix-cm}	
			\usepackage[english]{babel}
	
		%	=======================================================================================
		% 	tikz package
		% 	
		% 	--------------------------------- 	
			\usepackage{tikz}%
			\usetikzlibrary{arrows,positioning,shapes}
			\usetikzlibrary{mindmap}			
			

		% --------------------------------- 	page
			\usepackage{afterpage}		% 다음페이지가 나온면 어떻게 하라는 명령 정의 패키지
%			\usepackage{fullpage}			% 잘못 사용하면 다 흐트러짐 주의해서 사용
%			\usepackage{pdflscape}		% 
			\usepackage{lscape}			%	 


			\usepackage{blindtext}
	
		% --------------------------------- font 사용
			\usepackage{pifont}				%
			\usepackage{textcomp}
			\usepackage{gensymb}
			\usepackage{marvosym}



		% Package --------------------------------- 

			\usepackage{tablists}				%


		% Package --------------------------------- 
			\usepackage[framemethod=TikZ]{mdframed}				% md framed package
			\usepackage{smartdiagram}								% smart diagram package



		% Package ---------------------------------    연습문제 

			\usepackage{exsheets}				%

			\SetupExSheets{solution/print=true}
			\SetupExSheets{question/type=exam}
			\SetupExSheets[points]{name=point,name-plural=points}


		% --------------------------------- 페이지 스타일 지정

		\usepackage[Sonny]		{fncychap}

			\makeatletter
			\ChNameVar	{\Large\bf}
			\ChNumVar	{\Huge\bf}
			\ChTitleVar		{\Large\bf}
			\ChRuleWidth	{0.5pt}
			\makeatother

%		\usepackage[Lenny]		{fncychap}
%		\usepackage[Glenn]		{fncychap}
%		\usepackage[Conny]		{fncychap}
%		\usepackage[Rejne]		{fncychap}
%		\usepackage[Bjarne]	{fncychap}
%		\usepackage[Bjornstrup]{fncychap}

		\usepackage{fancyhdr}
		\pagestyle{fancy}
		\fancyhead{} % clear all fields
		\fancyhead[LO]{\footnotesize \leftmark}
		\fancyhead[RE]{\footnotesize \leftmark}
		\fancyfoot{} % clear all fields
		\fancyfoot[LE,RO]{\large \thepage}
		%\fancyfoot[CO,CE]{\empty}
		\renewcommand{\headrulewidth}{1.0pt}
		\renewcommand{\footrulewidth}{0.4pt}
	
	
	
		%	--------------------------------------------------------------------------------------- 
		% 	tritlesec package
		% 	
		% 	
		% 	------------------------------------------------------------------ section 스타일 지정
	
			\usepackage{titlesec}
		
		% 	----------------------------------------------------------------- section 글자 모양 설정
			\titleformat*{\section}					{\large\bfseries}
			\titleformat*{\subsection}				{\normalsize\bfseries}
			\titleformat*{\subsubsection}			{\normalsize\bfseries}
			\titleformat*{\paragraph}				{\normalsize\bfseries}
			\titleformat*{\subparagraph}				{\normalsize\bfseries}
	
		% 	----------------------------------------------------------------- section 번호 설정
			\renewcommand{\thepart}				{\arabic{part}.}
			\renewcommand{\thesection}				{\arabic{section}.}
			\renewcommand{\thesubsection}			{\thesection\arabic{subsection}.}
			\renewcommand{\thesubsubsection}		{\thesubsection\arabic{subsubsection}}
			\renewcommand\theparagraph 			{$\blacksquare$ \hspace{3pt}}

		% 	----------------------------------------------------------------- section 페이지 나누기 설정
			\let\stdsection\section
			\renewcommand\section{\newpage\stdsection}



		%	--------------------------------------------------------------------------------------- 
		% 	\titlespacing*{commandi} {left} {before-sep} {after-sep} [right-sep]		
		% 	left
		%	before-sep		:  수직 전 간격
		% 	after-sep	 	:  수직으로 후 간격
		%	right-sep

			\titlespacing*{\section} 			{0pt}{1.0em}{1.0em}
			\titlespacing*{\subsection}	  		{0ex}{1.0em}{1.0em}
			\titlespacing*{\subsubsection}		{0ex}{1.0em}{1.0em}
			\titlespacing*{\paragraph}			{0em}{1.5em}{1.0em}
			\titlespacing*{\subparagraph}		{4em}{1.0em}{1.0em}
	
		%	\titlespacing*{\section} 			{0pt}{0.0\baselineskip}{0.0\baselineskip}
		%	\titlespacing*{\subsection}	  		{0ex}{0.0\baselineskip}{0.0\baselineskip}
		%	\titlespacing*{\subsubsection}		{6ex}{0.0\baselineskip}{0.0\baselineskip}
		%	\titlespacing*{\paragraph}			{6pt}{0.0\baselineskip}{0.0\baselineskip}
	

		% --------------------------------- recommend		섹션별 페이지 상단 여백
		\newcommand{\SectionMargin}			{\newpage  \null \vskip 2cm}
		\newcommand{\SubSectionMargin}		{\newpage  \null \vskip 2cm}
		\newcommand{\SubSubSectionMargin}		{\newpage  \null \vskip 2cm}


		%	--------------------------------------------------------------------------------------- 
		% 	toc 설정  - table of contents
		% 	
		% 	
		% 	----------------------------------------------------------------  문서 기본 사항 설정
			\setcounter{secnumdepth}{4} 		% 문단 번호 깊이
			\setcounter{tocdepth}{2} 			% 문단 번호 깊이 - 목차 출력시 출력 범위

			\setlength{\parindent}{0cm} 		% 문서 들여 쓰기를 하지 않는다.


		%	--------------------------------------------------------------------------------------- 
		% 	mini toc 설정
		% 	
		% 	
		% 	--------------------------------------------------------- 장의 목차  minitoc package
			\usepackage{minitoc}

			\setcounter{minitocdepth}{1}    	%  Show until subsubsections in minitoc
%			\setlength{\mtcindent}{12pt} 	% default 24pt
			\setlength{\mtcindent}{24pt} 	% default 24pt

		% 	--------------------------------------------------------- part toc
		%	\setcounter{parttocdepth}{2} 	%  default
			\setcounter{parttocdepth}{0}
		%	\setlength{\ptcindent}{0em}		%  default  목차 내용 들여 쓰기
			\setlength{\ptcindent}{0em}         


		% 	--------------------------------------------------------- section toc

			\renewcommand{\ptcfont}{\normalsize\rm} 		%  default
			\renewcommand{\ptcCfont}{\normalsize\bf} 	%  default
			\renewcommand{\ptcSfont}{\normalsize\rm} 	%  default


		%	=======================================================================================
		% 	tocloft package
		% 	
		% 	------------------------------------------ 목차의 목차 번호와 목차 사이의 간격 조정
			\usepackage{tocloft}

		% 	------------------------------------------ 목차의 내어쓰기 즉 왼쪽 마진 설정
			\setlength{\cftsecindent}{2em}			%  section

		% 	------------------------------------------ 목차의 목차 번호와 목차 사이의 간격 조정
			\setlength{\cftsecnumwidth}{2em}		%  section





		%	=======================================================================================
		% 	flowchart  package
		% 	
		% 	------------------------------------------ 목차의 목차 번호와 목차 사이의 간격 조정
			\usepackage{flowchart}
			\usetikzlibrary{arrows}



		%	=======================================================================================
		% 	줄 간격 설정
		% 	
		% 	
		% 	--------------------------------- 	줄간격 설정
			\doublespace
%			\onehalfspace
%			\singlespace
		
		

	% 	============================================================================== itemi Global setting

	
		%	-------------------------------------------------------------------------------
		%		Vertical spacing
		%	-------------------------------------------------------------------------------
			\setlist[itemize]{topsep=0.0em}			% 상단의 여유치
			\setlist[itemize]{partopsep=0.0em}			% 
			\setlist[itemize]{parsep=0.0em}			% 
%			\setlist[itemize]{itemsep=0.0em}			% 
			\setlist[itemize]{noitemsep}				% 
			
		%	-------------------------------------------------------------------------------
		%		Horizontal spacing
		%	-------------------------------------------------------------------------------
			\setlist[itemize]{labelwidth=1em}			%  라벨의 표시 폭
			\setlist[itemize]{leftmargin=8em}			%  본문 까지의 왼쪽 여백  - 4em
			\setlist[itemize]{labelsep=3em} 			%  본문에서 라벨까지의 거리 -  3em
			\setlist[itemize]{rightmargin=0em}			% 오른쪽 여백  - 4em
			\setlist[itemize]{itemindent=0em} 			% 점 내민 거리 label sep 과 같은면 점위치 까지 내민다
			\setlist[itemize]{listparindent=3em}		% 본문 드려쓰기 간격
	
	
			\setlist[itemize]{ topsep=0.0em, 			%  상단의 여유치
						partopsep=0.0em, 		%  
						parsep=0.0em, 
						itemsep=0.0em, 
						labelwidth=1em, 
						leftmargin=2.5em,
						labelsep=2em,			%  본문에서 라벨 까지의 거리
						rightmargin=0em,		% 오른쪽 여백  - 4em
						itemindent=0em, 		% 점 내민 거리 label sep 과 같은면 점위치 까지 내민다
						listparindent=0em}		% 본문 드려쓰기 간격
	
%			\begin{itemize}
	
		%	-------------------------------------------------------------------------------
		%		Label
		%	-------------------------------------------------------------------------------
			\renewcommand{\labelitemi}{$\bullet$}
			\renewcommand{\labelitemii}{$\bullet$}
%			\renewcommand{\labelitemii}{$\cdot$}
			\renewcommand{\labelitemiii}{$\diamond$}
			\renewcommand{\labelitemiv}{$\ast$}		
	
%			\renewcommand{\labelitemi}{$\blacksquare$}   	% 사각형 - 찬것
%			\renewcommand\labelitemii{$\square$}		% 사각형 - 빈것	
			






% ------------------------------------------------------------------------------
% Begin document (Content goes below)
% ------------------------------------------------------------------------------
	\begin{document}
	
			\dominitoc
			\doparttoc			




			\title{터널}
			\author{김대희}
			\date{2017년 7월}
			\maketitle


			\tableofcontents 		% 목차 출력
%			\listoffigures 			% 그림 목차 출력
			\cleardoublepage
			\listoftables 			% 표 목차 출력





		\mdfdefinestyle	{con_specification} {
						outerlinewidth		=1pt			,%
						innerlinewidth		=2pt			,%
						outerlinecolor		=blue!70!black	,%
						innerlinecolor		=white 			,%
						roundcorner			=4pt			,%
						skipabove			=1em 			,%
						skipbelow			=1em 			,%
						leftmargin			=0em			,%
						rightmargin			=0em			,%
						innertopmargin		=2em 			,%
						innerbottommargin 	=2em 			,%
						innerleftmargin		=1em 			,%
						innerrightmargin		=1em 			,%
						backgroundcolor		=gray!4			,%
						frametitlerule		=true 			,%
						frametitlerulecolor	=white			,%
						frametitlebackgroundcolor=black		,%
						frametitleaboveskip=1em 			,%
						frametitlebelowskip=1em 			,%
						frametitlefontcolor=white 			,%
						}


	

%	================================================================== Part			골재
%	\addtocontents{toc}{\protect\newpage}
	\part{터널 지보재}
	\noptcrule
	\parttoc				


		

\newpage
\section[지보재 시공계획]{지보재 시공계획}

% -----------------------------------------------------------------------------
%
%
%
% -----------------------------------------------------------------------------


\subsection{일반사항}

\subsubsection{지보재 시공시기}
\begin{itemize}
\item 지보재는 굴착 후 가능한 조기에 설치하여 지반이완이 최소가 되도록 하여야 하며, 굴착면 지반의 자립시간 이내에 설치를 완료하여야 한다.
\item 지보재 공종 간에는 휴지시간이 없도록 설치시기를 계획하여야 하며 동일 작업조에 의하여 지보재 시공이 완료되도록 하여야 한다. 지보재 전공정 시공이 동일 작업조에 의해 수행될 수 없는 경우에는 공종별 작업조 간의 인수인계가 막장에서 이루어지도록 하여야 한다..
\end{itemize}


\subsubsection{지보재 시공순서 결정}
\begin{enumerate}
\item 지보재의 시공순서는 지반조건, 터널단면의 크기와 형상, 주변여건, 안정성 및 경제성 등을 분석하여 결정하여야 한다.
\item 지보재의 시공순서 변경이 요구되는 경우에는 지보재의 성능발생 특성에 따라 지반이 자체 지보능력을 발휘할 수 있도록 결정하여야 한다.  
\end{enumerate}

\subsubsection{지보재 시공 중 조치사항}
\begin{enumerate}
\item  시공 중 지보재 혹은 지반에 이상 변형이 발생하였을 경우에는 지체 없이 지보재의 강성 증가, 록볼트의 추가 타설, 링폐합 등의 보강을 시행하고 계측빈도를 증가시키고 필요시 추가의 계측기를 설치하여 변위추이를 감시하여야 한다.  
\item  시공 중 실제 지반조건이 설계 시 예측된 지반조건과 다를 경우에는 감독원의 승인을 얻은 후 지체 없이 실제 지반조건에 적합한 지보방법으로 변경하여 시공하여야 한다.  
\end{enumerate}

\subsection{재료}
해당사항 없음  

\subsection{시공}
해당사항 없음  

% -----------------------------------------------------------------------------
%
%
%
% -----------------------------------------------------------------------------

\newpage
\section{강지보재}

\subsection{일반사항}

\subsubsection{강지보재 일반}
\begin{enumerate}
\item 강지보재는 지반조건, 굴착방법, 굴착단면의 크기 등을 감안하여 제작하고 신속히 시공할 수 있도록 관리하여야 한다.  
\item 강지보재의 종류로는 구조용 H형강, U형강, 격자지보(lattice girder) 등이 있으며, 굴착면이나 숏크리트에 밀착되어 강지보재에 하중이 고루 분포되도록 하고, 콘크리트라이닝의 두께가 확보될 수 있도록 정확한 제작과 시공을 하여야 한다.  
\end{enumerate}


\subsection{재료}

\subsubsection{강지보재의 재질}
\begin{enumerate}
\item 강지보재는 연성이 크고 휨과 용접 등의 가공성이 양호한 강재를 사용하여야 한다.  
\item H형강과 U형강의 재질은 KS D3503에 규정된 SS 300B 또는 KS D 3515에 규정된 SM400, 격자지보의 재질은 KS D3504에 규정된 SD500W를 표준으로 하며 이와 동등 이상의 성능을 발휘하는 구조용 강재로 한다.  
\item 강재 대신 고강도 플라스틱, 복합부재 등을 지보재로 사용할 경우 강지보재와 동등 이상의 성능을 발휘하여야 한다.   
\end{enumerate}

\subsubsection{강지보재의 제작}
\begin{enumerate}
\item 강지보재는 작용하중, 숏크리트의 두께, 피복두께, 굴착공법 등을 고려하여 소요단면과 설계치수가 확보되도록 제작하여야 한다.  
\item 강지보재의 이음 개소는 거치 및 시공성을 고려하여 정하되 이음개소를 최소화하고 구조적으로 유리한 곳에서 견고하게 체결되도록 제작하여야 한다.  
\end{enumerate}

\subsubsection{재료의 품질관리}
\begin{enumerate}
\item 강지보재의 휨가공, 절단, 구멍내기, 용접 등의 가공방법과 형상, 크기 등은 공사시방서에 정하여 관리하여야 한다.  
\item 강지보재 재질의 품질관리는 공인기관의 시험성적을 확인하는 것으로 대신한다.  
\end{enumerate}


\newpage
\subsection{시공}

\subsubsection{강지보재의 설치}

\begin{enumerate}
\item 강지보재는 막장에 근접시켜 굴착 후 즉시 설치하여야 한다. \\ 단, 지반조건에 따라 강지보재를 생략할 수 있다.  
\item 강지보재의 설치간격은 \\ 지반특성, 사용목적, 시공법 등을 고려하여 조정할 수 있다.  
\item 강지보재 기초부에는 목재, 콘크리트 블록, 강판 등의 받침을 설치하여 강지보재가 견고하게 지지되도록 하여야 한다. 
      \\ 단, 목재는 굴착경계부와 같이 임시 받침으로만 사용될 수 있다.  
\item 강지보재 기초부에 전달되는 하중이 큰 경우에는 지지력을 확보할 수 있도록 \textbf {바닥보강 콘크리트 받침}을 사용하여야 한다.  
\item 굴착면이 튀어나와 강지보재의 설치가 곤란한 경우에는 튀어나온 부분을 제거한 후 설치하여야 하며 
      여굴에 의해 강지보재가 지반을 지지하지 못하게 되는 부분은 강재 또는 콘크리트 쐐기 등을 일정간격으로 설치하여 
      강지보재가 굴착면을 지지할 수 있도록 하여야 한다.  
\item 강지보재가 해체되거나 연결되는 곳에는 후속 시공이 용이하도록 조치를 취한 후 숏크리트를 타설하여야 한다.  
\item 강지보재 연결부는 강지보재 기능을 손상하지 않도록 체결하여야 하며, 강지보재 상호 간은 간격재로 견고하게 연결하여야 한다.  
\item 시공된 강지보재를 수정하여야 할 경우에는 1조 단위로 하여야 한다.  
\end{enumerate}


\subsubsection{현장 품질관리}

\begin{enumerate}
\item 강지보재의 품질관리는 〈표 5-2.1〉과 같이 관리하여야 한다.  
\end{enumerate}
   
   
표 5-2.1 강지보재의 현장 품질관리 항목

% -----------------------------------------------------------------------------
%
%
%
% -----------------------------------------------------------------------------


\newpage
\section[숏크리트]{숏크리트}

\subsection{일반사항}

\subsubsection{숏크리트 일반}

\begin{enumerate}
\item 숏크리트는 다음과 같은 기능을 발휘할 수 있도록 시공하여야 한다.
	\begin{enumerate}
	\item 숏크리트에 작용하는 외력을 지반에 분산시키고, 터널주변의 붕락하기 쉬운 암괴를 지지하며, 굴착면 가까이에 지반아치가 형성될 수 있도록 한다. 
	\item 강지보재 또는 록볼트에 지반압을 전달하는 기능을 발휘하도록 연속적으로 타설하여야 한다.
	\item 굴착된 지반의 굴곡부를 메우고 절리면 사이를 접착시킴으로써 응력집중현상을 피하도록 한다.
	\item 굴착면을 피복하여 풍화, 누수 및 세립자 유출 등을 방지하도록 한다.
	\end{enumerate}
\item 숏크리트는 사용수의 혼합방법에 따라 건식과 습식으로 구분하며 강(鋼) 또는 기타 재질의 섬유도 혼합하여 사용할 수 있다.
\item 숏크리트는 일반숏크리트와 고강도숏크리트로 구분하며 숏크리트의 설치목적에 맞게 성능과 품질을 확보하여야 한다. 
      숏크리트의 재령 28일 설계기준강도는 일반숏크리트의 경우 21MPa 이상, 고강도숏크리트의 경우 35MPa 이상이고 
      재령 1일 강도는 10MPa 이상이 되도록 하여야 한다.
\item 고강도숏크리트는 일반숏크리트 기능 이외에 다음과 같은 기능을 추가적으로 발휘할 수 있도록 하여야 한다.   
	\begin{enumerate}
	\item 고강도숏크리트 자체가 마감면이 되는 경우 화재에 대한 숏크리트의 안정성을 확보할 수 있는 대책을 세워야 한다.  
	\item 조명, 환기 등 기타 부대설비를 충분히 고정시킬 수 있을 만큼 숏크리트의 강도와 부착성능이 확보되어야 한다.
	\end{enumerate}
\item 고강도숏크리트가 최종 마감재로 사용되는 경우에는 소요 두께 및 층별 기능을 위해 각 층별로 다른 성능의 숏크리트를 타설할 수 있다.
\end{enumerate}

\newpage
\subsection{재료}

\subsubsection{숏크리트의 재질}
\begin{enumerate}
\item  2.1.1 일반사항  
	\begin{enumerate}
	\item  (1) 숏크리트의 성능은 터널용도에 맞는 뿜어붙이기 성능과 초기 및 장기강도, 내구성능을 설정하여야 한다. 
	\item  (2) 숏크리트의 뿜어붙이기 성능은 리바운드율, 분진 농도, 초기강도 및 장기강도로 설정할 수 있다. 
	\item  (3) 숏크리트에 섬유를 첨가한 섬유보강 숏크리트의 성능은 초기 및 장기강도 이외에 휨강도와 휨인성을 설정하여야 한다. 
	\end{enumerate}
\item  2.1.2 고강도숏크리트를 최종 마감재로 적용할 경우 구조적 안정성과 박락에 대한 저항성을 확보하기 위해 암반 및 숏크리트 각 층 간의 부착강도를 높일 필요가 있으며 재령 28일 부착강도는 1MPa 이상이 되도록 관리하여야 한다.  
\item  2.1.3 급결제는 요구되는 급결성을 갖는 액상이나 분말형을 사용할 수 있으며 조기 강도 발현 효과가 좋고 장기강도에 영향을 적게 미치는 것을 사용하여야 한다.   
\item  2.1.4 숏크리트용 시멘트는 KS L 5201(시멘트)의 기준에 적합한 1종의 보통 포틀랜드 시멘트 사용을 원칙으로 한다. 이 이외의 시멘트를 사용할 경우에는 규정된 숏크리트 성능기준 만족여부를 사전에 확인하여야 한다.  
\item  2.1.5 급결제 및 시멘트는 품질이 변질되지 않도록 보관하여야 한다. 동절기 및 혹서기에 골재가 동해되거나 가열되지 않도록 조치하여야 한다.  
\item  2.1.6 급결제는 KS F 2782의 규정에 적합하고 인체에 유해한 영향이 없어야 한다.  
\end{enumerate}

\subsubsection{숏크리트의 배합}
\begin{enumerate}
\item  2.2.1 굵은 골재 및 잔골재의 규격, 입도기준과 재료별 배합비율은 설계시방배합에 따라야 하며 현장배합시험 결과에 따라 조정하여야 한다.  
\item  2.2.2 재료별 배합은 중량배합으로 하여야 한다.  
\item  2.2.3 강섬유의 혼입량은 설계휨강도와 휨인성값이 만족될 수 있도록 배합관리하여야 하며 실제 벽면에 타설된 혼입량은 최소 295N/㎥(30kgf/㎥) 이상이 되어야 한다.  
\item  2.2.4 습식방식에 있어서 급결제 첨가 전의 베이스콘크리트는 굵은 골재의 최대치수, 슬럼프 및 배합강도를 기초로 정하며 베이스콘크리트를 펌프로 압송할 경우 슬럼프는 120mm 이상으로 한다.  
\end{enumerate}

\subsubsection{재료의 품질관리}
\begin{enumerate}
\item  2.3.1 재료는 요구되는 시험과 검사를 통하여 품질을 확인하여야 하며 시험에 대한 규정은 〈표 5-3.1〉을 따른다.  
\end{enumerate}

<table>
〈표 5-3.1〉 재료시험 사항
</table>

\subsubsection{철망}
\begin{enumerate}
\item 2.4.1 일반사항  
\begin{enumerate}
\item (1) 철망은 타설된 숏크리트가 자중으로 인해 박리될 가능성 등이 있는 경우에 숏크리트의 인장강도 및 전단강도의 향상, 숏크리트의 부착력 향상, 그리고 분할굴착 시 발생하는 시공이음부의 보강 등을 위하여 사용한다. 
\item (2) 강섬유보강 숏크리트를 사용할 경우는 철망을 생략할 수 있다. 
\end{enumerate}
\item 2.4.2 재료  
\begin{enumerate}
\item (1) 철망은 KS D 7017에 규정된 용접철망을 사용하되 철망의 지름은 5mm 내외, 개구 크기는 100mm×100mm 또는 150mm×150mm를 표준으로 한다.  
\item (2) 굴착면의 자립이 어렵고 숏크리트 타설 시 숏크리트 박리가 발생하는 경우에는 숏크리트와 지반과의 부착을 증진시키기 위하여 개구 크기와 철선지름이 작은 철망을 사용할 수 있다. 
\end{enumerate}
\item 2.4.3 시공 및 품질관리  
	\begin{enumerate}
	\item (1) 철망은 굴착면 또는 이미 타설된 숏크리트면에 밀착시켜 견고하게 고정하여 숏크리트 작업 중 이동이나 진동이 발생하지 않도록 하여야 한다. 
	\item (2) 철망은 터널 종방향으로 100mm, 횡방향으로 200mm 정도 중첩하여 이음하여야 한다. 
	\item (3) 철망은 보관 및 운반 시에 변형, 유해한 녹, 기타의 이물질이 부착되지 않도록 조치하여야 한다. 
	\item (4) 철망의 품질관리는 〈표 5-3.2〉와 같이 하여야 한다. 
	\end{enumerate}
\end{enumerate}
 
표 5-3.2〉 철망의 현장 품질관리 사항

\newpage
\subsection{시공}

\subsubsection{사전준비 및 처리}
\begin{enumerate}
\item  3.1.1 굴착면으로부터 뜬 돌을 주의하여 제거하여야 하며 숏크리트의 부착을 저해하는 요인들을 제거하여야 한다.  
\item  3.1.2 잔골재는 호스가 폐쇄되지 않고 먼지의 발생이 적도록 표면수를 함유하여야 한다.  
\item  3.1.3 굴착면이나 이미 타설한 숏크리트면에 용출수가 있을 경우에는 용출수 대책을 강구한 후 숏크리트를 타설하여야 한다.  
\item  3.1.4 용출수 대책으로는 배수관을 이용한 배수, 시멘트량이나 급결재량의 증가, 사용수량의 감소 등의 방법을 적용할 수 있다.  
\end{enumerate}

\subsubsection{숏크리트 타설기계 및 타설방법}
\begin{enumerate}
\item  3.2.1 숏크리트 타설기계는 내압에 대한 안전성, 내구성은 물론 양호한 기계적 특성 보유 유무, 시공조건 등을 검토하여 소정의 배합재료를 연속하여 압송할 수 있는 것을 선정하여야 한다.  
\item  3.2.2 숏크리트 타설기계는 굴착면 인접부까지 접근이 가능하여야 하며 요구되는 기능을 발휘할 수 있는 부속기기를 갖추어야 한다.  
\item  3.2.3 숏크리트 타설방법과 공법은 지반조건, 터널연장 및 단면크기 굴착방법, 용출수의 유무, 경제성 등을 검토하여 변경할 수 있다.   
\end{enumerate}

\subsubsection{숏크리트 작업}
\begin{enumerate}
\item  3.3.1 숏크리트는 굴착 후 조속히 시공하여야 하며, 지반과 밀착되도록 타설하여야 한다.   
\item  3.3.2 강지보재, 철망, 철근 등이 있는 경우에는 숏크리트와 강지보재가 일체가 되도록 주의해서 뿜어붙여야 한다.  
\item  3.3.3 숏크리트를 타설한 후 저온, 건조, 급격한 온도변화 등 해로운 영향을 받지 않도록 보호하고 양생을 하여야 한다.  
\item  3.3.4 숏크리트 기계작업원과 타설작업원 간의 거리는 상호 수신호가 가능한 거리 이내이어야 한다.  
\item  3.3.5 숏크리트의 타설작업 시에는 철망, 철근, 강지보재 등의 배면에 공동이나 틈이 발생되지 않도록 하여야 하며, 철망, 철근은 숏크리트 타설로 인하여 이동과 과도한 진동 등이 생기지 않도록 고정하여야 한다.  
\item  3.3.6 숏크리트를 나누어 타설하는 경우에는 숏크리트 각층이 상호 확실히 부착되도록 타설하여야 한다.  
\item  3.3.7 숏크리트 타설 시 반발된 숏크리트가 혼합되지 않도록 반발된 숏크리트는 모두 제거하여야 한다.  
\item  3.3.8 시공된 숏크리트면은 평탄하게 하되 각 경우별로 평탄성의 허용치를 설정하여 관리할 수 있다.  
\item  3.3.9 노즐의 방향은 숏크리트면에 직각이 되도록 유지하고 굴착면과의 거리는 반발량이 최소화되도록 유지하여야 한다.  
\item  3.3.10 숏크리트의 타설작업은 하부로부터 상부로 진행하되 강지보재 부분을 먼저 타설하여 강지보재와 숏크리트의 일체성을 증진하여야 한다.  
\item  3.3.11 건식타설 방법에 있어서 물의 압력은 압축공기의 압력보다 100kPa 정도 높게 되도록 유지하여야 한다.  
\item  3.3.12 숏크리트 타설작업원은 골재의 반발이나 분진의 위해가 있을 경우에 대비하여 보호장비를 착용하여야 한다.  
\end{enumerate}

\subsubsection{분진 및 반발량 처리}
\begin{enumerate}
\item  3.4.1 숏크리트 타설작업장은 분진처리를 하여야 하며 양호한 작업환경을 유지하여야 한다.   
\item  3.4.2 숏크리트 타설 시 발생된 반발재는 굳기 전에 제거하여야 한다.  
\end{enumerate}

\subsubsection{현장 품질관리}
\begin{enumerate}
\item  3.5.1 숏크리트는 설계기준 강도가 발휘되도록 하고 반발률을 적게 하며 양호한 작업성을 갖도록 관리하여야 한다.  
\item  3.5.2 숏크리트 품질관리는 〈표 5-3.3〉과 같이 하여야 한다.  
 
   
〈표 5-3.3〉 숏크리트의 현장 품질관리 사항
 
\begin{enumerate}
\item  (1) 숏크리트의 두께는 시공 시에는 핀 등을 이용하여 측정하고 정기관리를 위해서는 천공하여 측정하여야 한다. 
\item  (2) 숏크리트의 두께는 설계두께를 기준으로 하여 검측된 평균두께가 설계두께 이상이어야 하며 검측된 최소두께는 설계두께의 75\% 이상이어야 한다. 
\item  (3) 숏크리트의 두께측정 결과두께가 설계두께에 미달되는 구간은 좌우 1m 범위 내에서 재측정하여 상기(2)항과 같은 기준으로 판정하고 재측정결과 판정기준에 미달하면 표본면적으로 대표된 전면적을 설계두께 이상으로 보완하여야 하며 보완시공의 최소두께는 30mm 이상으로 하여야 한다. 
\end{enumerate}
 
\item 3.5.4 숏크리트의 강도시험은 다음의 규정에 따라야 한다.  
\begin{enumerate}
	\item 시공 전의 강도시험용 시료성형은 \\ 콘크리트의 휨강도시험용 몰드(150×150 ×530mm KS F 2422)를 사용하여 반발재가 유출되도록 70° 정도 경사지게 한 후 뿜어붙인다. \\뿜어붙인 후 몰드의 윗부분을 조심하여 평탄하게 고르고 압축강도시험 및 휨강도시험용의 공시체로 사용하도록 한다. 
\item 시공 후 압축강도시험용의 시료채취는 \\ 숏크리트 타설 후 28일 경과 시 코어보링머신을 이용하여 Φ100×200mm 또는 Φ50×100mm의 규격으로 1회 시험당 3개씩의 원주형 시료를 채취하고 시료의 양 단면은 콘크리트 절삭기 등으로 직각으로 절단하거나 KS F 2403에 준하여 캡핑하여야 한다. 
\item 숏크리트의 시간에 따른 강도발현 상태를 파악하기 위하여 단기 및 장기재령 강도시험을 시행하여야 하며 단기재령 강도시험은 24시간, 장기재령 강도는 28일로 하여야 한다. 숏크리트 기준강도는 Φ150×300mm 규격의 성형시료를 기준으로 한 값이며, 시험시료의 크기가 달라질 경우에는 보정하여야 한다.  
\item 현장채취코어를 통한 숏크리트의 압축강도 및 강섬유보강 숏크리트의 시험은 1회 시험당 3개의 코어시료를 채취하고, 채취한 모든 시료의 시험강도는 설계강도의 75\%보다 작아서는 안 되며, 산술평균강도는 설계강도의 85\% 이상이어야 한다. 또한 시료를 성형하는 경우는 3개의 시험결과 중 2개 이상은 설계강도 이상이어야 하며 나머지 1개는 설계강도의 85\%보다 작아서는 안 되며, 평균강도는 설계강도 이상이어야 한다. 
\item 숏크리트의 단기 및 장기재령 강도시험에서 미달되는 경우는 재료의 변경 또는 현장배합설계를 조정하여 설계강도를 확보할 수 있도록 하여야 한다. 
\item 시공된 숏크리트의 재령 28일 압축강도가 1차 시험에서 미달되는 구간은 좌우 5m 범위 내에서 재시험용 코어를 채취하여 상기(4)항과 같은 기준으로 판정하고 재시험결과 판정기준에 미달 시에는 보완시공 또는 재시공하여야 한다. 
\end{enumerate}
\item  3.5.5 숏크리트의 반발률 측정은 현장에서 숏크리트를 타설하고 바닥에 떨어진 숏크리트(반발재)를 수거, 계량하여 다음 식에 의하여 반발률을 산출한다.  
\item  3.5.6 상기와 같은 현장 품질관리시험이 종료되면 시공자는 시험결과를 각 항목별로 일정양식에 정리하여야 한다.  
\item  3.5.7 강섬유혼입량시험은 터널 내에 시공된 숏크리트에서 직접 코어(Φ100mm 이상)를 채취한 후 KS F 2781 숏크리트용 강섬유보강 콘크리트의 강섬유 혼입률 시험방법에 따라 시험을 실시한다. 단, 적용된 숏크리트가 목표로 하는 휨강도, 휨인성을 만족할 경우 혼입량 검사를 생략할 수 있다.   
\end{enumerate}

% ----------------------------------------------------------------------------------
%
%
%
% ----------------------------------------------------------------------------------
\newpage
\section{록볼트}

\subsection{일반사항}

\subsubsection{록볼트 일반}
\begin{enumerate}
\item  록볼트는 굴착면 주변에 설치되어 이완된 암괴 또는 절리면들을 꿰메어 지반을 보강하거나 내압을 가하여 지반이 아치나 보를 
       형성하도록 시공하여야 한다.  
\item  록볼트 설치는 지반조건을 분석하여 록볼트의 효과와 기능을 발휘할 수 있는지 여부를 검토하여 소요의 기능을 발휘할 수 있는 
       록볼트를 시공하여야 한다.  
\item  록볼트의 인장력을 발휘하도록 시공할 경우에는 발생축력을 검토하여 록볼트의 재질과 형상을 결정하고 인발내력을 확인하여야 한다.  
\item  록볼트의 재질, 지압판, 정착형식 및 정착재료의 선정 등에 있어서는 그 시공성을 검토하여야 한다.  
\item  록볼트에 프리스트레스를 도입할 경우의 프리스트레스는 \textbf {록볼트 항복응력의 80\% 이하} 
       가 되도록 하여야 한다.  
\item  록볼트 천공 시 지반조건상 공벽의 자립성이 불량한 경우에는 자천공 록볼트를 사용할 수 있다.  
\item  8m 이상의 긴 록볼트를 설치할 필요가 있는 경우에는 일반 록볼트와 함께 긴 케이블볼트를 조합하여 사용할 수 있다.  
\end{enumerate}

\subsection{재료}

\subsubsection{록볼트의 재료}
\begin{enumerate}
\item  록볼트의 재질은 SD350 이상의 강재로서 연신율이 큰 재질이어야 하며, 이형봉강이나 강관, 팽창성 강관 등을 사용할 수 있다.  
\item  현장조건 및 시공여건에 따라 섬유 또는 유리재질로 보강된 소재의 록볼트를 사용할 수 있다.  
\item  지압판의 두께는 6mm를 표준으로 하되 팽창성 지반에 대해서는 9mm 이상의 두께를 사용하여야 한다.  
\item  케이블볼트의 재질 및 형상은 원지반 조건 및 사용목적에 따라 정하여야 하며, 
       일반적으로 재질은 공칭지름 12.7mm 이상의 7연선으로 인장강도 및 연신율이 큰 것이어야 하고 
       1본의 케이블볼트가 지탱하는 소요강도에 따라 다양한 형상의 케이블볼트를 사용할 수 있다.  
\end{enumerate}

\subsubsection{록볼트의 정착형식}
\begin{enumerate}
\item  록볼트의 정착형식은 선단정착형, 전면접착형, 혼합형으로 구분한다.  
\item  선단정착형 록볼트는 \\ 록볼트의 선단을 지반에 정착한 후 프리스트레스를 도입하여 굴착면 주변지반에 내압이 작용하도록 하여야 한다.  
\item  전면접착형 록볼트는 \\ 수지 또는 모르터 등을 사용하여 록볼트 전장을 지반에 정착시키는 형식으로서 
       굴착면 주변지반의 지보능력을 향상시키도록 하여야 한다.  
\item  혼합형 록볼트는 \\ 록볼트 선단을 지반에 정착시키고 프리스트레스를 도입한 후 록볼트 전장과 지반과의 공극을 
       정착재료로 충전하는 형식으로서 선단정착형식과 전면접착형식의 기능을 모두 발휘할 수 있도록 하여야 한다.  
\end{enumerate}

\subsection{시공}

\subsubsection{천공기계의 선정}
\begin{enumerate}
\item  천공기계는 지반조건, 터널단면의 크기와 형상, 연장, 굴착공법, 천공길이, 본수 등을 고려하여 선정하여야 한다.  
\item  천공 도중 천공각도를 일정하게 유지시킬 수 있는 기계를 선정하여야 한다.  
\item  록볼트의 삽입, 정착, 조이기 등에 사용하는 기계는 록볼트의 정착형식에 적합한 것을 선정하여야 한다.  
\end{enumerate}

\subsubsection{  천공 및 청소}
\begin{enumerate}
\item  록볼트 천공은 소정의 위치, 직경, 깊이를 준수하여 굴착면에 직각이 되도록 천공하여야 한다. 
       \\ 단, 암괴의 탈락이 예상되는 주절리면이 파악된 경우에는 절리면의 직각방향으로 추가공을 천공하도록 한다.  
\item  록볼트 삽입 전 천공된 구멍에 돌가루 등의 록볼트 정착에 유해한 물질이 남지 않도록 청소하여야 한다.  
\item  록볼트는 삽입 전에 유해한 녹, 기타의 이물질이 부착되지 않도록 관리하여야 한다.  
\end{enumerate}

\subsubsection{정착재료 및 충전}
\begin{enumerate}
\item  록볼트의 정착재료는 유동성 및 접착성이 우수하고 조강성을 가지며 장기 안정성이 있는 것이라야 한다.  
\item  록볼트는 소정의 깊이까지 삽입하여야 하며 소정의 정착력을 얻도록 시공하여야 한다.  
\item  전면접착형 록볼트는 천공구멍과 록볼트 사이의 공극에 정착재가 완전히 채워져 록볼트가 소정의 정착력을 발휘할 수 있도록 하여야 한다.  
\item  록볼트 정착재로서 시멘트 모르터 등을 사용할 경우에는 다음의 규정을 따라야 한다.  
	\begin{enumerate}
	\item  시멘트는 보통 포틀랜드 시멘트를 사용한다. 조기에 접착능력을 발휘하여야 할 경우에는 급결제 등을 혼합하거나 
	       조강 시멘트를 사용하여야 한다. 
	\item  사용하는 모래는 최대 직경이 2mm 이하인 입도가 양호한 모래를 사용하여야 한다. 
	\item  시멘트와 모래의 배합은 1:1로 한다. 
	\item  물과 시멘트의 비는 40~50\%이며 플로우값은 200~220 정도를 기준으로 한다. 
	       단, 지하수 및 지반조건에 따라 물의 양을 가감하여 시공 중 시멘트 모르터의 유동성을 유지하여야 한다. 
	\item  시멘트 모르터의 충전은 모르터피더를 이용하거나 캡슐형을 사용할 수 있다. 
	\end{enumerate}
\item  현장여건상 록볼트 정착재료로서 시멘트 풀을 사용할 경우에는 시멘트 모르터와 동등 이상의 성능을 발휘할 수 있어야 한다.  
\item  록볼트 정착재료로서 레진(resin)을 사용할 경우는 다음의 규정을 따라야 한다.  
	\begin{enumerate}
	\item  레진은 폴리에스터(polyester)계 및 동등 이상의 재질이어야 하며 캡슐형태로 제공되어야 한다. 
	\item  레진은 제조업자가 표시한 보관기간이 경과하거나 변질 또는 용량이 불충분한 것을 사용하지 않도록 하여야 한다. 
	\item  용수, 염수, 산, 약 알카리성에 대하여 영향을 받지 않아야 하며 보관 및 이동에서 영상 50℃ 이상 
	       및 영하 30℃ 이하의 조건이 발생되지 않도록 하	여야 하며 색깔이 변질된 것을 사용하여서는 안 된다. 
	\item  레진의 인발내력은 규정된 록볼트의 허용인장강도보다 1.2배 이상 이어야 하며 조기에 접착력을 발휘하여야 한다. 
	\item  레진에 대한 현장 품질관리의 일반시험으로서 소정의 설계 인발력 이상이 됨을 확인하여야 한다. 
	       단, 토사 및 완전 풍화된 암반구간에는 레진을 사용하지 않도록 하여야 한다. 
	\end{enumerate}
\item  록볼트 정착재료 충전 시 소정의 인발내력과 장기내구성 확보를 위하여 정착재료가 흘러내리지 않도록 해야 한다.   
\end{enumerate}

\subsubsection{록볼트 조이기}
\begin{enumerate}
\item  3.4.1 록볼트의 조이기는 록볼트의 항복강도를 넘지 않는 범위 내에서 견고하게 조여야 한다.  
\item  3.4.2 전면접착형 록볼트는 정착 후 지압판 등이 숏크리트면에 밀착되도록 너트 등으로 조여야 한다.  
\item  3.4.3 선단정착형 록볼트의 경우에는 확고한 정착부를 형성하도록 하여 프리스트레스 도입 후에도 록볼트에 긴장응력이 유지되어야 한다. 지반조건상 시간이 경과하며 정착부가 느슨해질 우려가 있거나 록볼트의 부식이 크게 우려되는 곳에는 프리스트레스 도입 후 록볼트와 지반 사이의 공극을 시멘트 풀 혹은 모르터로 충전하여야 한다.  
\item  3.4.4 프리스트레스 록볼트의 경우는 록볼트 조이기를 실시한 후 1일 정도 후에 다시 조여야 한다. 또 그 후에도 정기적으로 점검하여, 소요의 긴장력이 도입되어 있는지를 확인하고, 이완되어 있는 경우에는 다시 조여야 한다.  
\end{enumerate}

\subsubsection{용출수지역에서의 록볼트 시공}
\begin{enumerate}
\item  3.5.1 용출수가 있을 경우 원칙적으로 용출수를 처리한 후 록볼트를 시공하여야 한다.  
\item  3.5.2 용출수로 인해 록볼트 충전이 어려운 경우는 급결제 등을 사용하거나 팽창성 강관 록볼트 등을 사용할 수 있다.  
\end{enumerate}

\subsubsection{  현장 품질관리}
\begin{enumerate}
\item  3.6.1 록볼트의 현장 품질관리는 〈표 5-4.1〉과 같이 하여야 한다.  
〈표 5-4.1〉 록볼트의 현장 품질관리
\item  3.6.2 인발시험을 실시할 록볼트는 시험구간 내에서 임의로 선택하며 선정된 록볼트의 지압판은 록볼트의 축과 직각을 이루도록 석고 또는 모르터 등으로 처리하여야 한다.  
\item  3.6.3 인발시험은 충분한 정착효과가 얻어진 후에 실시하여야 하며 인발하중의 재하속도는 분당 10kN 내외로 하여야 한다.  
\item  3.6.4 인발시험은 하중 단계별로 변위를 측정하여 하중변위곡선을 작성하고 판정 시의 변위가 설계에서 고려한 록볼트의 효과를 발휘할 수 있는 범위 이내인지를 확인하여 합격여부를 판정하여야 한다.  
\item  3.6.5 록볼트 인발시험은 사용된 록볼트와 동종인 록볼트 자체에 대한 사전인발시험을 실시하여 인발내력을 확인하고, 시공된 록볼트에 대한 실제 시험 시에는 설계인발내력의 80\%에 달하면 합격하는 것으로 한다.  
\item  3.6.6 인발시험 결과 불합격될 경우는 불합격 부위(천장, 아치, 측벽)에서 5개를 추가 시행하여 5개 중 3개 이상이 불합격되면 표본구간으로 대표된 전구간에 대하여 설계와 동일수량의 록볼트를 재시공하여야 한다.  
\item  3.6.7 다음의 경우는 록볼트를 추가 시공하여야 한다.  
\item  (1) 내공변위 또는 지중변위 측정 등의 계측결과로부터 얻어진 터널측면의 변형이 과다한 경우 
\item  (2) 록볼트 두부로부터 록볼트 길이의 1/2지점과 록볼트 단부 사이에 록볼트 축력의 최대치가 발생하는 경우 
\item  (3) 소성영역의 확대가 록볼트 길이를 넘는다고 판단되는 경우 
\item  (4) 숏크리트의 균열, 지압판의 변형, 과다변위, 응력발생 등 록볼트의 추가시공이 필요하다고 판단되는 경우 
 
\item  3.6.8 지반조건이 매우 불량하여 특정구간 전체에서 소요의 인발 내력이 얻어지지 않거나 록볼트 시공이 곤란한 경우에는 감독원과 협의하여 다른 조치를 강구할 수 있다.  
\item  3.6.9 록볼트에 대한 현장 품질관리시험이 종료되면 시공자는 시험결과를 각 항목별로 일정양식에 정리하여야 한다.  
\end{enumerate}









%	================================================================== Part			골재
%	\addtocontents{toc}{\protect\newpage}
	\part{터널 계측}
	\noptcrule
	\parttoc				

		
%	================================================================== Part			골재
%	\addtocontents{toc}{\protect\newpage}
%	\part{터널 계측}
%	\noptcrule
%	\parttoc				




% ================================================= chapter 	====================
	\chapter{과업개요}
	\minitoc				% Creating an actual minitoc


	% ------------------------------------------ section ------------ 
	\section{과업 개요}

		\paragraph{과업명}
		
		\paragraph{과업위치}
		
		\paragraph{과업목적}
		
		\paragraph{과업구간}
		
		\paragraph{과업개요}
		
		\paragraph{과업기간}
		
		
		%	-----------------------------------------------------------  section  
	\section{계측계획서의 제출}


		\begin{mdframed}[style=con_specification, frametitle={\large 일반국도공사 전문시방서}]
계약 상대자는 터널공사를 착수하기 전에 계측기의 정밀도 $\cdot$ 측정범위 $\cdot$ 작동 온도범위 등을 포함한 계측기의 제원, 계측기의 보정방법 및 유지관리, 계측결과의 기록양식 $\cdot$ 분석범위 $\cdot$ 분석방법 및 활용방안, 계측값의 수렴여부 판정방법 등을 포함하는 계측계획서를 작성하여 감독자에게 제출하여야 한다.
		\end{mdframed}

		

% ================================================= chapter 	====================
	\chapter{계측관리계획}
	\minitoc				% Creating an actual minitoc
		

	% ------------------------------------------ section ------------ 
	\section{계측항목별 설계수량}
		
		
	% ------------------------------------------ section ------------ 
	\section{계측관리 업무 협력 체계}
		
		
	% ------------------------------------------ section ------------ 
	\section{계측관리 인원조직도 및 장비 투입 계획}
		
		\paragraph{현장 투입 인원}


		\paragraph{장비 투입 게획}
		

		\paragraph{자재 투입 계획}
		
	%	------------------------------------------------------------------------------  table

			\begin{table} [h]
	
			\caption{자재 투입 계획} 
			\label{tab:title} 
	
			\begin{center}
			\tabulinesep=0.4em
			\begin{tabu} to 1.0\linewidth { X[1.4,r] X[1.4,l] X[0.6,c] X[c] X[0.6,c] }
			\tabucline [1pt,] {-}
			품명			&규격				&단위		&수량		&비고\\
			\tabucline [0.1pt,] {-}
			천단침하 타켓	&					&조			&181		&\\
			내공변위 타켓	&					&조			&664		&\\
			지표침하		&					&조			&12			&\\
			지중침하		&Vibrationg Wire	&조			&12			&\\
			선지보축력		&Vibrationg Wire	&조			&12			&\\
			지중변위		&Vibrationg Wire	&조			&50			&\\
			숏크리트 응력	&Vibrationg Wire	&조			&50			&\\
			락볼트축력		&Vibrationg Wire	&조			&50			&\\
			\tabucline [0.1pt,] {-}
			\end{tabu} 
			\end{center}
			\end{table}
		




% ------------------------------------------ section ------------ 
	\section{계측 시공 관리 계획}

		\paragraph{계측항목}


		\begin{description}
		\item[천단침하계]  Halpo
		\item[Email Address:] halpo@users.mysite.com
		\item[Address:]  1234 Ivy Ln \\ Springfield, USA
		\end{description}
		
		\paragraph{계측기 설치 위치 및 수량}
		
		\paragraph{계측빈도}
		
		\paragraph{계측기 설치 및 측정 방법}
		



% ================================================= chapter 	====================
	\chapter{계측 수행 및 결과 보고}
	\minitoc				% Creating an actual minitoc

% ------------------------------------------ section ------------ 
	\section{계측 수행 및 자료정리}
	
		\paragraph{계측수행}
	
		\paragraph{계측자료 정리}

% ------------------------------------------ section ------------ 
	\section{계측 결과 보고}
	
		\paragraph{계측 결과 보고 체계}
		
		\paragraph{계측 결과 보고서 제출}
		
		
% ================================================= chapter 	====================
	\chapter{계측 관리 기준}
	\minitoc				% Creating an actual minitoc
		
% ------------------------------------------ section ------------ 
	\section{계측 관리 기준}
	
% ------------------------------------------ section ------------ 
	\section{터널구간 계측관리체제}
			
% ------------------------------------------ section ------------ 
	\section{시공단계별 계측치 활용방안}


		\paragraph{계측에 의한 시공 평가 흐름도}
		
		\paragraph{계측 결과에 따른 합리적인 시공방안}

% ------------------------------------------ section ------------ 
	\section{일반적인 계측관리 기준}

		\paragraph{터널구간에서의 계측 관리기준}
		
		\paragraph{서울시 지하철 공사에 있어서의 관리기준}
		
		\paragraph{천단침하의 관리기준치}

		\paragraph{천단침하의 관리기준치 (프랑스 공업성기준)}
		
		\paragraph{천단침하의 관리기준치 (일본토질공학회지기준)}
		
		\paragraph{내공변위의 관리기준치}
		

		\paragraph{B계측 (R/B축력, S/C응력, 지중변위) 관리기준치)}
		
		
		
% ------------------------------------------ section ------------ 
	\section{당 현장 계측관리기준}
			
		\paragraph{당현장 계측관리 기준}
		
		\paragraph{일상계측 (천단 및 내공변위계)}
		
		\paragraph{정밀계측 (지중변위계, 숏크리트 응력계, 락볼트 축력계)}

		\paragraph{정밀계측 (지표침하계, 지중침하계, 선지보축력계)}


% ================================================= chapter 	====================
	\chapter{계측 품질관리 계획}
	\minitoc				% Creating an actual minitoc
	
% ------------------------------------------ section ------------ 
	\section{터널시공중의 현상 및 대응대책}
	
% ------------------------------------------ section ------------ 
	\section{막장안정 대책}
	
% ------------------------------------------ section ------------ 
	\section{관리기준치 초과시 조치요령}

		\paragraph{1차 관리기준치 초과시}
	
		\paragraph{2차 관리기준치 초과시}

		\paragraph{3차 관리기준치 초과시}

% ================================================= chapter 	====================
	\chapter{안전관리추진계획}
	\minitoc				% Creating an actual minitoc
	
% ------------------------------------------ section ------------ 
	\section{안전교육}
	
% ------------------------------------------ section ------------ 
	\section{현장지원계획}


% ================================================= chapter 	====================
	\chapter{부록}
	\minitoc				% Creating an actual minitoc

% ------------------------------------------ section ------------ 
	\section{계측계획도 및 상세도}

% ------------------------------------------ section ------------ 
	\section{예정공정표}
	










% ------------------------------------------------------------------------------
% End document
% ------------------------------------------------------------------------------
\end{document}



% =================================================================================================== Part 혼화 재료

% ========================================================================================= chapter

%	-----------------------------------------------------------  section  

	%	------------------------------------------------------------------------------  table

			\begin{table} [h]
	
			\caption{잔 골재의 표준입도}  
			\label{tab:title} 
	
			\begin{center}
			\tabulinesep=0.4em
			\begin{tabu} to 0.8\linewidth { X[r] X[l] X[c]  }
			\tabucline [1pt,] {-}
			체의 호칭 치수 (mm)		& 체를 통화한 것의 질량 백분율(\%) \\
			\multicolumn{4}{c} {단위량} \\
			\tabucline [0.1pt,] {-}
			2.5	&100\\
			1.2	& 99 $\sim$ 100 \\
			0.6	& 60 $\sim$ 80 \\
			0.3	& 20 $\sim$ 50 \\
			0.15	&  5 $\sim$ 30 \\
			\tabucline [0.1pt,] {-}
			\end{tabu} 
			\end{center}
			\end{table}



	%	------------------------------------------------------------------------------  문제

		\clearpage
		\begin{small}	
		\begin{question}
		레디 믹스트 콘크리트에서 \textbf{회수수}를 혼합수로 사용할 경우 주의할 사항 중 틀린것은 ?
		\begin{enumerate}[label=\arabic*), topsep=0.0em, itemsep=-1.0em ]
			\item [①] 고강도 콘크리트의 경우 회수수를 사용하여서는 안된다. 
			\item [②] 슬러지수의 사용 시 \textbf{단위 슬러지 고형분}은 콘크리트 질량의 3\% 이하로 한다. 
			\item [③] 회수수의 품질 시험 항목은 4가지로 염소 이온량, 시멘트 응결 시간의 차, 모르타르 압축강도의 비, 단위 슬러지 고형분율 이다. 
			\item [④] 콘크리트를 배합할 때, 회수수 중에 함유된 슬러지 고형분은 물의 질량에는 포함되지 않는다. 
		\end{enumerate}
		\end{question}



		\begin{solution}
		해설
		\end{solution}
		\end{small}	
		\hrulefill


		%	----------------------------------------------------------------------------- 수식
			\begin{equation}
			\begin{aligned}
			T_2 = T_1 - 0.15 ( T_1 - T_0 ) \times t
			\end{aligned}
			\end{equation}


			\begin{description}[style=sameline, leftmargin=2cm, topsep=0.0em, itemsep=0.0em]
			\item[$T_0$] 		주의의 온도(${}^\circ C$)
			\item[$T_1$] 		비볐을때의 콘크리트 온도 (${}^\circ C$)
			\item[$t$] 		비빈후 부터 타설이 끝났을때 까지의 시간 (h)
			\end{description}



