
%	================================================================== Part			골재
%	\addtocontents{toc}{\protect\newpage}
%	\part{터널 계측}
%	\noptcrule
%	\parttoc				




% ================================================= chapter 	====================
	\chapter{과업개요}
	\minitoc				% Creating an actual minitoc


	% ------------------------------------------ section ------------ 
	\section{과업 개요}

		\paragraph{과업명}
		
		\paragraph{과업위치}
		
		\paragraph{과업목적}
		
		\paragraph{과업구간}
		
		\paragraph{과업개요}
		
		\paragraph{과업기간}
		
		
		%	-----------------------------------------------------------  section  
	\section{계측계획서의 제출}


		\begin{mdframed}[style=con_specification, frametitle={\large 일반국도공사 전문시방서}]
계약 상대자는 터널공사를 착수하기 전에 계측기의 정밀도 $\cdot$ 측정범위 $\cdot$ 작동 온도범위 등을 포함한 계측기의 제원, 계측기의 보정방법 및 유지관리, 계측결과의 기록양식 $\cdot$ 분석범위 $\cdot$ 분석방법 및 활용방안, 계측값의 수렴여부 판정방법 등을 포함하는 계측계획서를 작성하여 감독자에게 제출하여야 한다.
		\end{mdframed}

		

% ================================================= chapter 	====================
	\chapter{계측관리계획}
	\minitoc				% Creating an actual minitoc
		

	% ------------------------------------------ section ------------ 
	\section{계측항목별 설계수량}
		
		
	% ------------------------------------------ section ------------ 
	\section{계측관리 업무 협력 체계}
		
		
	% ------------------------------------------ section ------------ 
	\section{계측관리 인원조직도 및 장비 투입 계획}
		
		\paragraph{현장 투입 인원}


		\paragraph{장비 투입 게획}
		

		\paragraph{자재 투입 계획}
		
	%	------------------------------------------------------------------------------  table

			\begin{table} [h]
	
			\caption{자재 투입 계획} 
			\label{tab:title} 
	
			\begin{center}
			\tabulinesep=0.4em
			\begin{tabu} to 1.0\linewidth { X[1.4,r] X[1.4,l] X[0.6,c] X[c] X[0.6,c] }
			\tabucline [1pt,] {-}
			품명			&규격				&단위		&수량		&비고\\
			\tabucline [0.1pt,] {-}
			천단침하 타켓	&					&조			&181		&\\
			내공변위 타켓	&					&조			&664		&\\
			지표침하		&					&조			&12			&\\
			지중침하		&Vibrationg Wire	&조			&12			&\\
			선지보축력		&Vibrationg Wire	&조			&12			&\\
			지중변위		&Vibrationg Wire	&조			&50			&\\
			숏크리트 응력	&Vibrationg Wire	&조			&50			&\\
			락볼트축력		&Vibrationg Wire	&조			&50			&\\
			\tabucline [0.1pt,] {-}
			\end{tabu} 
			\end{center}
			\end{table}
		




% ------------------------------------------ section ------------ 
	\section{계측 시공 관리 계획}

		\paragraph{계측항목}


		\begin{description}
		\item[천단침하계]  Halpo
		\item[Email Address:] halpo@users.mysite.com
		\item[Address:]  1234 Ivy Ln \\ Springfield, USA
		\end{description}
		
		\paragraph{계측기 설치 위치 및 수량}
		
		\paragraph{계측빈도}
		
		\paragraph{계측기 설치 및 측정 방법}
		



% ================================================= chapter 	====================
	\chapter{계측 수행 및 결과 보고}
	\minitoc				% Creating an actual minitoc

% ------------------------------------------ section ------------ 
	\section{계측 수행 및 자료정리}
	
		\paragraph{계측수행}
	
		\paragraph{계측자료 정리}

% ------------------------------------------ section ------------ 
	\section{계측 결과 보고}
	
		\paragraph{계측 결과 보고 체계}
		
		\paragraph{계측 결과 보고서 제출}
		
		
% ================================================= chapter 	====================
	\chapter{계측 관리 기준}
	\minitoc				% Creating an actual minitoc
		
% ------------------------------------------ section ------------ 
	\section{계측 관리 기준}
	
% ------------------------------------------ section ------------ 
	\section{터널구간 계측관리체제}
			
% ------------------------------------------ section ------------ 
	\section{시공단계별 계측치 활용방안}


		\paragraph{계측에 의한 시공 평가 흐름도}
		
		\paragraph{계측 결과에 따른 합리적인 시공방안}

% ------------------------------------------ section ------------ 
	\section{일반적인 계측관리 기준}

		\paragraph{터널구간에서의 계측 관리기준}
		
		\paragraph{서울시 지하철 공사에 있어서의 관리기준}
		
		\paragraph{천단침하의 관리기준치}

		\paragraph{천단침하의 관리기준치 (프랑스 공업성기준)}
		
		\paragraph{천단침하의 관리기준치 (일본토질공학회지기준)}
		
		\paragraph{내공변위의 관리기준치}
		

		\paragraph{B계측 (R/B축력, S/C응력, 지중변위) 관리기준치)}
		
		
		
% ------------------------------------------ section ------------ 
	\section{당 현장 계측관리기준}
			
		\paragraph{당현장 계측관리 기준}
		
		\paragraph{일상계측 (천단 및 내공변위계)}
		
		\paragraph{정밀계측 (지중변위계, 숏크리트 응력계, 락볼트 축력계)}

		\paragraph{정밀계측 (지표침하계, 지중침하계, 선지보축력계)}


% ================================================= chapter 	====================
	\chapter{계측 품질관리 계획}
	\minitoc				% Creating an actual minitoc
	
% ------------------------------------------ section ------------ 
	\section{터널시공중의 현상 및 대응대책}
	
% ------------------------------------------ section ------------ 
	\section{막장안정 대책}
	
% ------------------------------------------ section ------------ 
	\section{관리기준치 초과시 조치요령}

		\paragraph{1차 관리기준치 초과시}
	
		\paragraph{2차 관리기준치 초과시}

		\paragraph{3차 관리기준치 초과시}

% ================================================= chapter 	====================
	\chapter{안전관리추진계획}
	\minitoc				% Creating an actual minitoc
	
% ------------------------------------------ section ------------ 
	\section{안전교육}
	
% ------------------------------------------ section ------------ 
	\section{현장지원계획}


% ================================================= chapter 	====================
	\chapter{부록}
	\minitoc				% Creating an actual minitoc

% ------------------------------------------ section ------------ 
	\section{계측계획도 및 상세도}

% ------------------------------------------ section ------------ 
	\section{예정공정표}
	




